\documentclass[11pt]{exam}
\newcommand{\myname}{Jackson Eshbaugh}
\newcommand{\myemail}{eshbaugj}
\newcommand{\myhwtype}{MSG Activity}
\newcommand{\myhwnum}{1}
\newcommand{\myclass}{CS 104}
\newcommand{\mylecture}{1}
\newcommand{\mysection}{2}

% Prefix for numedquestion's
\newcommand{\questiontype}{Exercise}

% Use this if your "written" questions are all under one section
% For example, if the homework handout has Section 5: Written Questions
% and all questions are 5.1, 5.2, 5.3, etc. set this to 5
% Use for 0 no prefix. Redefine as needed per-question.
\newcommand{\writtensection}{0}

\usepackage{amsmath, amsfonts, amsthm, amssymb}  % Some math symbols
\usepackage{enumerate}
\usepackage{enumitem}
\usepackage{graphicx}
\usepackage{hyperref}
\usepackage[all]{xy}
\usepackage{wrapfig}
\usepackage{fancyvrb}
\usepackage[T1]{fontenc}
\usepackage{float}
\usepackage{listings}
\usepackage{booktabs}
\usepackage{framed}
\usepackage{parcolumns}

\usepackage{centernot}
\usepackage{mathtools}
\DeclarePairedDelimiter{\ceil}{\lceil}{\rceil}
\DeclarePairedDelimiter{\floor}{\lfloor}{\rfloor}
\DeclarePairedDelimiter{\card}{\vert}{\vert}

% Uncomment the following line to get Solarized-themed source listings
% You will have had to already installed the solarized-light package
% https://github.com/jez/latex-solarized
%
%\usepackage{solarized-light}

\setlength{\parindent}{0pt}
\setlength{\parskip}{5pt plus 1pt}
\pagestyle{empty}

\def\indented#1{\list{}{}\item[]}
\let\indented=\endlist

\newcounter{questionCounter}
\newcounter{partCounter}[questionCounter]

\newenvironment{namedquestion}[1][\arabic{questionCounter}]{%
    \addtocounter{questionCounter}{1}%
    \setcounter{partCounter}{0}%
    \vspace{.2in}%
    \noindent{\bf #1}%
    \vspace{0.3em} \hrule \vspace{.1in}%
}{}

\newenvironment{numedquestion}[0]{%
    \stepcounter{questionCounter}%
    \vspace{.2in}%
    \ifx
        \writtensection\undefined
        \noindent{\bf \questiontype \; \arabic{questionCounter}. }%
    \else
        \if
            \writtensection0
            \noindent{\bf \questiontype \; \arabic{questionCounter}. }%
        \else
            \noindent{\bf \questiontype \; \writtensection.\arabic{questionCounter} }%
        \fi
        \vspace{0.3em} \hrule \vspace{.1in}%
        }{}

\newenvironment{alphaparts}[0]{%
    \begin{enumerate}[label=\textbf{(\alph*)}]
    }{\end{enumerate}}

\newenvironment{arabicparts}[0]{%
    \begin{enumerate}[label=\textbf{\arabic{questionCounter}.\arabic*})]
    }{\end{enumerate}}

\newenvironment{questionpart}[0]{%
    \item
    }{}

\newcommand{\answerbox}[1]{
    \begin{framed}
    \vspace{#1}
    \end{framed}}

\pagestyle{head}

\headrule
\header{\textbf{\myclass\ \mylecture\mysection}}%
{\textbf{\myname\ }}%
{\textbf{\myhwtype\ \myhwnum}}

\begin{document}
    \thispagestyle{plain}
    \begin{center}
    {\Large \myclass{} \myhwtype{} \myhwnum}:
        \\
        {\Large Variables \& Data Types}
        \\
        \myname{}\\
        \today
    \end{center}

    \begin{numedquestion}
    	How would you define the term \textit{variable}?
    	
    	\answerbox{3cm}
    \end{numedquestion}
    
	% model an array    
    
    \begin{numedquestion}
    	What does the memory look like after the following code runs? \\
        \begin{verbatim}
int i = 32;
int j = 2;
int k = i + j;
int l = i * j;
        \end{verbatim}
    	\answerbox{3cm}
    \end{numedquestion}
    
    % matrix (lin alg)
    % print contents of array
    \pagebreak
    \begin{numedquestion}
    	Complete the right side of the list:
        \begin{table}[h]
            \centering
            \begin{tabular}{|c|c|}
                \hline
                                \textbf{Data Type} & \textbf{Represents} \\ \hline
                                \texttt{int} & \hspace{5cm} \\ \hline
                                \texttt{float} & \hspace{5cm} \\ \hline
                                \texttt{double} & \hspace{5cm} \\ \hline
                                \texttt{char} & \hspace{5cm} \\ \hline
                                \texttt{boolean} & \hspace{5cm} \\ \hline
                                \texttt{String} & \hspace{5cm} \\ \hline
                            \end{tabular}
            \caption{Data Types and What They Represent}
            \label{tab:data_types}
        \end{table}
    \end{numedquestion}

    \begin{numedquestion}
        What will the following code output? What will the memory look like after execution?
    \begin{verbatim}


                                int i = 12;
                                int j = 7;
                                println(3 * x);
                                int k = i + j - 2;
                                println(k);

                                int l = 3;
                                println(l / 2);
    \end{verbatim}

    \answerbox{4cm}

    \end{numedquestion}

    \begin{numedquestion}
        Solve the following:

        \begin{verbatim}
15 % 5 =               8 % 3 =               3 % 2 =               12 % 15 =
        \end{verbatim}
    \end{numedquestion}
\end{document}