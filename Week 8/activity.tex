\documentclass[11pt]{exam}
\newcommand{\myname}{Jackson Eshbaugh}
\newcommand{\myemail}{eshbaugj}
\newcommand{\myhwtype}{MSG Activity}
\newcommand{\myhwnum}{7}
\newcommand{\myclass}{CS 104}
\newcommand{\mylecture}{1}
\newcommand{\mysection}{2}

% Prefix for numedquestion's
\newcommand{\questiontype}{Exercise}

% Use this if your "written" questions are all under one section
% For example, if the homework handout has Section 5: Written Questions
% and all questions are 5.1, 5.2, 5.3, etc. set this to 5
% Use for 0 no prefix. Redefine as needed per-question.
\newcommand{\writtensection}{0}

\usepackage{amsmath, amsfonts, amsthm, amssymb}  % Some math symbols
\usepackage{enumerate}
\usepackage{enumitem}
\usepackage{graphicx}
\usepackage{hyperref}
\usepackage[all]{xy}
\usepackage{wrapfig}
\usepackage{fancyvrb}
\usepackage[T1]{fontenc}
\usepackage{float}
\usepackage{listings}
\usepackage{booktabs}
\usepackage{framed}
\usepackage{parcolumns}

\usepackage{centernot}
\usepackage{mathtools}
\DeclarePairedDelimiter{\ceil}{\lceil}{\rceil}
\DeclarePairedDelimiter{\floor}{\lfloor}{\rfloor}
\DeclarePairedDelimiter{\card}{\vert}{\vert}

% Uncomment the following line to get Solarized-themed source listings
% You will have had to already installed the solarized-light package
% https://github.com/jez/latex-solarized
%
%\usepackage{solarized-light}

\setlength{\parindent}{0pt}
\setlength{\parskip}{5pt plus 1pt}
\pagestyle{empty}

\def\indented#1{\list{}{}\item[]}
\let\indented=\endlist

\newcounter{questionCounter}
\newcounter{partCounter}[questionCounter]

\newenvironment{namedquestion}[1][\arabic{questionCounter}]{%
    \addtocounter{questionCounter}{1}%
    \setcounter{partCounter}{0}%
    \vspace{.2in}%
    \noindent{\bf #1}%
    \vspace{0.3em} \hrule \vspace{.1in}%
}{}

\newenvironment{numedquestion}[0]{%
    \stepcounter{questionCounter}%
    \vspace{.2in}%
    \ifx
        \writtensection\undefined
        \noindent{\bf \questiontype \; \arabic{questionCounter}. }%
    \else
        \if
            \writtensection0
            \noindent{\bf \questiontype \; \arabic{questionCounter}. }%
        \else
            \noindent{\bf \questiontype \; \writtensection.\arabic{questionCounter} }%
        \fi
        \vspace{0.3em} \hrule \vspace{.1in}%
        }{}

\newenvironment{alphaparts}[0]{%
    \begin{enumerate}[label=\textbf{(\alph*)}]
    }{\end{enumerate}}

\newenvironment{arabicparts}[0]{%
    \begin{enumerate}[label=\textbf{\arabic{questionCounter}.\arabic*})]
    }{\end{enumerate}}

\newenvironment{questionpart}[0]{%
    \item
    }{}

\newcommand{\answerbox}[1]{
    \begin{framed}
    \vspace{#1}
    \end{framed}}

\pagestyle{head}

\headrule
\header{\textbf{\myclass\ \mylecture\mysection}}%
{\textbf{\myname\ }}%
{\textbf{\myhwtype\ \myhwnum}}

\begin{document}
    \thispagestyle{plain}
    \begin{center}
    {\Large \myclass{} \myhwtype{} \myhwnum}:
        \\
        {\Large 2-Dimensional Arrays}
        \\
        \myname{}\\
        Week 8
    \end{center}
    Given the array declaration below, complete the following problems. \\
    \texttt{float[][] gradebook = new float[25][4]}

    \begin{numedquestion}
        Suppose that in \texttt{gradebook[i][j]}, \texttt{i} is the student ID and \texttt{j} is the assignment (0 is Midterm 1, 1 is Lab Exam, 2 is Midterm 2, and 3 is Final Exam).
        How could you compute the average grade on Midterm 1?
        \begin{answerbox}{2in}
        \end{answerbox}
    \end{numedquestion}

    \begin{numedquestion}
      Suppose that in \texttt{gradebook[i][j]}, \texttt{i} is the student ID and \texttt{j} is the assignment (0 is Midterm 1, 1 is Lab Exam, 2 is Midterm 2, and 3 is Final Exam). How could you compute the average grade on an given assignment index \texttt{i}?
        \begin{answerbox}{2in}
        \end{answerbox}
    \end{numedquestion}
\newpage
    \begin{numedquestion}
        Suppose the class is graded in a total points fashion. Write a function \texttt{finalGrade(float[][] gradebook, int id)} to compute the final grade of a given student with ID \texttt{id}.
        \begin{answerbox}{3in}
        \end{answerbox}
    \end{numedquestion}

    \begin{numedquestion}
       Suppose the class is graded in a total points fashion. Write a function \texttt{finalGrades(float[][] gradebook)} that returns a (one-dimensional) array of student final grades. You are encouraged to use the \texttt{finalGrade(float[][] gradebook, int id)} function you created earlier.
        \begin{answerbox}{3in}
        \end{answerbox}
    \end{numedquestion}
\end{document}