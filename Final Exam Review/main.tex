\documentclass[11pt]{article}
\usepackage[margin=1in]{geometry}
\usepackage{fancyhdr}
\usepackage{amsmath, amssymb}
\usepackage{enumerate}
\usepackage{listings}
\usepackage{titlesec}

\pagestyle{fancy}
\fancyhf{}
\rhead{CS 104 — Review Worksheet}
\lhead{Jackson Eshbaugh}
\cfoot{\thepage}

\titleformat{\section}{\normalfont\Large\bfseries}{}{0pt}{}

\lstset{
  basicstyle=\ttfamily\small,
  frame=single,
  breaklines=true
}

\begin{document}

\begin{center}
  {\LARGE \textbf{Final Exam Review}}\\[1ex]
  {\large CS 104: Intro to Game Programming}\\
  \today
\end{center}

\vspace{1em}

\section{Functions}

\textbf{1. Label the Function}: In the below code, label the parts of the function (fill in the blanks), using their general terms (i.e., don't add any types like \texttt{int}).
\vspace{1em}
\begin{lstlisting}[language=Java]

_____________________  _________________(______ _______, _______ ________) {
  
  _____
  
}

\end{lstlisting}

\noindent\textbf{2. Practice}: Write a function \texttt{int countPositive(int[] nums)} that returns the count of elements in \texttt{nums} that are larger than 0.

\vspace{15em}

\pagebreak

\section{Loops}
\textbf{1. Converting Loops}: Convert the following loop to a \texttt{for} loop.

\begin{lstlisting}[language=Java]
int x = 0;
while(x <= 3) {
  println(x);
  x++;
}

// The equivalent for loop:











//
\end{lstlisting}

\vspace{1em}

\section{Arrays}

\textbf{1. Brainstorm}: Why might we use (or have we used) arrays?

\vspace{5em}

\noindent\textbf{2. Scenario}: Suppose you're a biology student looking to write a program to compute basic statistics for your lab.

\begin{enumerate}[(a)]
	\item Write a function that computes the variance of an array of floats, given \( x_1, x_2, \dots, x_n \) (the array of values) and \( \bar{x} \) (the mean). You need to compute:
	\[ s^2 = \frac{\sum[(x_i - \bar{x})^2]}{n-1} \]
	\vspace{25em}
	\item Write another function that calculates the standard deviation, given the same as in \textit{(a)}. You are encourage to use the function you defined in \textit{(a)}. Assume the function \texttt{sqrt()} takes the square root. Your function needs to compute:
	
	\[ \text{standard deviation} = s = \sqrt{s^2} \]
	\vspace{15em}
\end{enumerate}

\section{Objects}

\textbf{1. Classes \& Polymorphism}: Complete the following parts.

\begin{enumerate}[(a)]
	\item Create a class \texttt{Person} to hold general information about a person. Create an instance of this class.
	\vspace{25em}
	\pagebreak
	\item Create a \underline{subclass} of \texttt{Person} called \texttt{Sudent} that holds information about a student. Create a new \texttt{Person} variable and assign a new \texttt{Student} to it.
	\vspace{30em}
	\item Create another subclass of \texttt{Person} called \texttt{Faculty} that holds information about a faculty member. Create a \texttt{Person} array of length 3 and assign the \texttt{Person} from part \textit{(a)} to the first slot in the array and the \texttt{Student} from part \textit{(b)} to the second slot. Finally, add a new \texttt{Faculty} object to the last slot.
\end{enumerate}

\pagebreak

\section{Java}
\textbf{1. Translate}: Translate the below Processing code into proper Java code.

\begin{lstlisting}[language=Java]
void setup() {
  println("Hello, World!");
}

// The Java code for this:


















//
\end{lstlisting}

\end{document}