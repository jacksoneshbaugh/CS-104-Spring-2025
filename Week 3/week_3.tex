\documentclass[11pt]{exam}
\usepackage{listings}
\usepackage{color}
\newcommand{\myname}{Jackson Eshbaugh}
\newcommand{\myemail}{eshbaugj}
\newcommand{\myhwtype}{MSG Activity}
\newcommand{\myhwnum}{2}
\newcommand{\myclass}{CS 104}
\newcommand{\mylecture}{1}
\newcommand{\mysection}{2}

% Prefix for numedquestion's
\newcommand{\questiontype}{Exercise}

% Use this if your "written" questions are all under one section
% For example, if the homework handout has Section 5: Written Questions
% and all questions are 5.1, 5.2, 5.3, etc. set this to 5
% Use for 0 no prefix. Redefine as needed per-question.
\newcommand{\writtensection}{0}

\usepackage{amsmath, amsfonts, amsthm, amssymb}  % Some math symbols
\usepackage{enumerate}
\usepackage{enumitem}
\usepackage{graphicx}
\usepackage{hyperref}
\usepackage[all]{xy}
\usepackage{wrapfig}
\usepackage{fancyvrb}
\usepackage[T1]{fontenc}
\usepackage{float}
\usepackage{listings}
\usepackage{booktabs}
\usepackage{framed}
\usepackage{parcolumns}

\usepackage{centernot}
\usepackage{mathtools}
\DeclarePairedDelimiter{\ceil}{\lceil}{\rceil}
\DeclarePairedDelimiter{\floor}{\lfloor}{\rfloor}
\DeclarePairedDelimiter{\card}{\vert}{\vert}

% Uncomment the following line to get Solarized-themed source listings
% You will have had to already installed the solarized-light package
% https://github.com/jez/latex-solarized
%
%\usepackage{solarized-light}

\setlength{\parindent}{0pt}
\setlength{\parskip}{5pt plus 1pt}
\pagestyle{empty}

\def\indented#1{\list{}{}\item[]}
\let\indented=\endlist

\newcounter{questionCounter}
\newcounter{partCounter}[questionCounter]

\newenvironment{namedquestion}[1][\arabic{questionCounter}]{%
    \addtocounter{questionCounter}{1}%
    \setcounter{partCounter}{0}%
    \vspace{.2in}%
    \noindent{\bf #1}%
    \vspace{0.3em} \hrule \vspace{.1in}%
}{}

\newenvironment{numedquestion}[0]{%
    \stepcounter{questionCounter}%
    \vspace{.2in}%
    \ifx
        \writtensection\undefined
        \noindent{\bf \questiontype \; \arabic{questionCounter}. }%
    \else
        \if
            \writtensection0
            \noindent{\bf \questiontype \; \arabic{questionCounter}. }%
        \else
            \noindent{\bf \questiontype \; \writtensection.\arabic{questionCounter} }%
        \fi
        \vspace{0.3em} \hrule \vspace{.1in}%
        }{}

\newenvironment{alphaparts}[0]{%
    \begin{enumerate}[label=\textbf{(\alph*)}]
    }{\end{enumerate}}

\newenvironment{arabicparts}[0]{%
    \begin{enumerate}[label=\textbf{\arabic{questionCounter}.\arabic*})]
    }{\end{enumerate}}

\newenvironment{questionpart}[0]{%
    \item
    }{}

\newcommand{\answerbox}[1]{
    \begin{framed}
    \vspace{#1}
    \end{framed}}

\pagestyle{head}

\headrule
\header{\textbf{\myclass\ \mylecture\mysection}}%
{\textbf{\myname\ }}%
{\textbf{\myhwtype\ \myhwnum}}

\definecolor{codegray}{rgb}{0.5,0.5,0.5}
\definecolor{codepurple}{rgb}{0.58,0,0.82}
\definecolor{backcolour}{rgb}{0.95,0.95,0.92}

\lstdefinestyle{mystyle}{
    backgroundcolor=\color{backcolour},   
    commentstyle=\color{codegray},
    keywordstyle=\color{magenta},
    numberstyle=\tiny\color{codegray},
    stringstyle=\color{codepurple},
    basicstyle=\ttfamily\footnotesize,
    breakatwhitespace=false,         
    breaklines=true,                 
    captionpos=b,                    
    keepspaces=true,                 
    numbers=left,                    
    numbersep=5pt,                  
    showspaces=false,                
    showstringspaces=false,
    showtabs=false,                  
    tabsize=2
}

\lstset{style=mystyle}

\begin{document}
    \thispagestyle{plain}
    \begin{center}
    {\Large \myclass{} \myhwtype{} \myhwnum}:
        \\
        {\Large Variables \& Scoping}
        \\
        \myname{}\\
        \today
    \end{center}

\begin{numedquestion}
    Describe the difference between local and global variables.
    \answerbox{2in}
\end{numedquestion}

\begin{numedquestion}
    What is the lifetime of a variable?
    \answerbox{2in}
\end{numedquestion}
\pagebreak
\begin{numedquestion}
    Given the following code, what will print?
    \begin{lstlisting}[language=Java]
    int x = 10;

    void setup() {
        int x = 5;
        println(x);
    }

    void draw() {
        println(x);
    }
    \end{lstlisting}
    \answerbox{1in}
\end{numedquestion}

\begin{numedquestion}
    What is the scope of each variable?
    \begin{lstlisting}[language=Java]
    int a = 20;

    void setup() {
        int b = 15;
        println(b);
    }

    void draw() {
        println(a);
    }
    \end{lstlisting}
    \answerbox{1.5in}
\end{numedquestion}

\begin{numedquestion} 
    What will happen when you run this code? If there is an error, explain why and fix it.
    \begin{lstlisting}[language=Java]
    void setup() {
        println(num);
    }

    void draw() {
        int num = 100;
        println(num);
    }
    \end{lstlisting}
    \answerbox{1.5in}
\end{numedquestion}

\begin{numedquestion} 
    What will be printed in the console when you run this code?
    \begin{lstlisting}[language=Java]
    int counter = 0;

    void setup() {
        counter = counter + 5;
        println(counter);
    }

    void draw() {
        int counter = 10;
        counter = counter + 1;
        println(counter);
    }
    \end{lstlisting}

    \answerbox{1in}
    \textbf{Challenge:}  
    Modify the code so that \texttt{counter} always increases globally and does not reset inside \texttt{draw()}.
\end{numedquestion}

\end{document}
